Programa suskaičiuoja studento vidurkį arba medianą.

{\bfseries{Programos veikimas\+:}} Su v0.\+1\+:
\begin{DoxyItemize}
\item įvesti vardą ir pavardę
\item įvesti arba sugeneruoti namų darbų pažymius bei egzamino balą
\item suskaičiuoja suvestų pažymių vidurkį arba mediana
\end{DoxyItemize}

Su v0.\+2\+:
\begin{DoxyItemize}
\item pasirinkti nuskaityti arba įvesti duomenis pačiam
\item nuskaityti duomenis iš failo
\item duomenys yra suruošiuojami pagal pavardes
\item vidurkis ir mediana apvalinami iki sviekosios dalies
\end{DoxyItemize}

Su v0.\+3
\begin{DoxyItemize}
\item funkcijos perkeltos į naują \char`\"{}function.\+cpp\char`\"{} failą
\item sukurtas išimpčių valdymas, kuris tikrina ar nuskaitomi tesingi pažymiai bei ar skaitomas failas egzistuoja
\end{DoxyItemize}

Su v0.\+4
\begin{DoxyItemize}
\item programa nieko vartotojo nebeklausia
\item sukuriamas failas
\item nuskaitomas failas
\item studentai surušiuojami į dvi grupes pagal galutinį pažymį $<$5\+: \char`\"{}vargsiukai\char`\"{}, $>$=5\+: \char`\"{}galvociai\char`\"{}
\item skaičiuojamas bendras programos ir kiekvienos funkcijos veikimo laikas bei atspausdinamas
\end{DoxyItemize}

Su v0.\+5 (Konteinerių testavimas)
\begin{DoxyItemize}
\item programa papildomai parašyta naudojant std\+::list ir std\+::deque
\item testuotjamas nuskaitymo ir rūšiavimo į dvi grupes greitis
\end{DoxyItemize}

Su v1.\+0 (Rūšiavimo optimizavimas)
\begin{DoxyItemize}
\item std\+::vector, std\+::deque ir std\+::list konteineriams rūšiavimo funkcijai pritaikytos dvi strategijos
\item testuojamas dviejų strategijų nuskaitymo, rūšiavimo į dvi grupes ir bendras programos greitis
\end{DoxyItemize}

Strategijos\+:
\begin{DoxyItemize}
\item 1 strategija\+: Bendro studentai konteinerio (vector, list ir deque tipų) skaidymas (rūšiavimas) į du naujus to paties tipo konteinerius\+: \char`\"{}vargšiukų\char`\"{} ir \char`\"{}kietiakų\char`\"{}. Tokiu būdu tas pats studentas yra dvejuose konteineriuose\+: bendrame studentai ir viename iš suskaidytų (vargšiukai arba kietiakai).
\item 2 strategija\+: Bendro studentų konteinerio (vector, list ir deque) skaidymas (rūšiavimas) panaudojant tik vieną naują konteinerį\+: \char`\"{}vargšiukai\char`\"{}. Tokiu būdu, jei studentas yra vargšiukas, jį turime įkelti į naująjį \char`\"{}vargšiukų\char`\"{} konteinerį ir ištrinti iš bendro studentai konteinerio. Po šio žingsnio studentai konteineryje liks vien tik kietiakai.
\end{DoxyItemize}

Pritaikytų strategijų rezultatai\+:  Rezultatas\+: 1 strategijoje greičiau veikia vector, o deque ir list sparta priklauso nuo studentų sarašo dydžio. 2 strategijoje vector ir deque naudojant vector\+::erase ir deque\+::erase programa žymiai lėčiau veikia, nei list, kurioje irgi buvo naudota list\+::erase.

2 strategiją pakeičiau naudojant remove\+\_\+if, kuriuo iš sarašo ištryniau studentus, kurių pažymys $<$5\+: 

Rezultatas\+: std\+::vector pakoreguota 2 strategija ir vietoj vector\+::erase panaudota std\+::remove\+\_\+if bei tai gerokai paspartino rūšiavimo greitį.

Su v1.\+1
\begin{DoxyItemize}
\item vietoj vektoriaus struktūros naudojama vectoriaus klasė
\item palyginimi programos veikimo laikai su ankščiau kurta std\+::vector struktūra
\item panaudojami optimizavimo flag\textquotesingle{}ai, palyginami rezultatai
\end{DoxyItemize}

\tabulinesep=1mm
\begin{longtabu}spread 0pt [c]{*{6}{|X[-1]}|}
\hline
\PBS\centering \cellcolor{\tableheadbgcolor}\textbf{ Versija   }&\PBS\centering \cellcolor{\tableheadbgcolor}\textbf{ Naudojamas tipas   }&\PBS\centering \cellcolor{\tableheadbgcolor}\textbf{ Flag\textquotesingle{}as   }&\PBS\centering \cellcolor{\tableheadbgcolor}\textbf{ 100000 studentų failas   }&\PBS\centering \cellcolor{\tableheadbgcolor}\textbf{ 1000000 studentų failas   }&\PBS\centering \cellcolor{\tableheadbgcolor}\textbf{ Failo dydis    }\\\cline{1-6}
\endfirsthead
\hline
\endfoot
\hline
\PBS\centering \cellcolor{\tableheadbgcolor}\textbf{ Versija   }&\PBS\centering \cellcolor{\tableheadbgcolor}\textbf{ Naudojamas tipas   }&\PBS\centering \cellcolor{\tableheadbgcolor}\textbf{ Flag\textquotesingle{}as   }&\PBS\centering \cellcolor{\tableheadbgcolor}\textbf{ 100000 studentų failas   }&\PBS\centering \cellcolor{\tableheadbgcolor}\textbf{ 1000000 studentų failas   }&\PBS\centering \cellcolor{\tableheadbgcolor}\textbf{ Failo dydis    }\\\cline{1-6}
\endhead
v1.\+0   &struktūra   &nėra   &2.\+1747   &20.\+9054   &188KB    \\\cline{1-6}
v1.\+1   &struktūra   &O1   &2.\+0345   &20.\+0918   &184KB    \\\cline{1-6}
v1.\+1   &struktūra   &O2   &2.\+1033   &19.\+5852   &184KB    \\\cline{1-6}
v1.\+1   &struktūra   &O3   &2.\+0104   &19.\+5326   &184KB    \\\cline{1-6}
v1.\+1   &klasė   &nėra   &2.\+0273   &22.\+1732   &178KB    \\\cline{1-6}
v1.\+1   &klasė   &O1   &2.\+1664   &21.\+0803   &178KB    \\\cline{1-6}
v1.\+1   &klasė   &O2   &2.\+2216   &20.\+8976   &178KB    \\\cline{1-6}
v1.\+1   &klasė   &O3   &2.\+1694   &20.\+8815   &178KB   \\\cline{1-6}
\end{longtabu}


Su v1.\+2
\begin{DoxyItemize}
\item sukurtas vienas seteris visoms reikšmėms perduoti
\item sukurtas konstruktorius ir desturktorius
\end{DoxyItemize}

Su v1.\+5
\begin{DoxyItemize}
\item Sukurtos dvi klasės\+: bazinė \mbox{\hyperlink{class_zmogus}{Zmogus}} klasė bendrai aprašyti žmogų ir iš jos išvestinė (derived) klasė -\/ \mbox{\hyperlink{class_studentas}{Studentas}}
\item Žmogui skirta bazinė klasė yra abstrakčioji klasė, t.\+y. negalima sukurti žmogaus tipo objektų, o tik objektus gautus iš jos išvestinių klasiu
\end{DoxyItemize}

Sy v2.\+0
\begin{DoxyItemize}
\item Sukurta dokumentacija panaudojant Doxygen
\item Atlikti Unit testai
\item Sukurtas automatinis įdiegimo failas
\end{DoxyItemize}

{\bfseries{Išimčių valdymas\+:}}\textbackslash{} Jeigu neranda skaitymo failo gaunama žinutė\+: 
\begin{DoxyCode}{0}
\DoxyCodeLine{Nerastas failas!}

\end{DoxyCode}
 Blogas pažymio ir egzamino įvedimas faile išmeta\+: 
\begin{DoxyCode}{0}
\DoxyCodeLine{Netinkamas namu darbu pazymis skaitymo failel! Pazymis turi buti >=0 \&\& <=10!}
\DoxyCodeLine{Netinkamas egzamino pazymis skaitymo failel! Pazymis turi buti >=0 \&\& <=10!}

\end{DoxyCode}
 Programos sukurtos naudojant std\+::vector (\char`\"{}vector\char`\"{}), std\+::list (\char`\"{}list\char`\"{}), std\+::deque (\char`\"{}deque\char`\"{}).

{\bfseries{Įdiegimo instrukcija\+:}}\textbackslash{} Atsisiųsti \+\_\+release ir jį išarchivuoti. Terminale nuėjus į išarchivuotą aplanką atlikti\+:
\begin{DoxyItemize}
\item Kompiliavimui atitinkamai naudoti make vector/deque/list komandą\+:\textbackslash{}
\item .o failo ištrinimui naudoti make clean komandą.
\item Paleisti ./\mbox{[}pavadinimas\mbox{]}\textbackslash{} Makefile\+: 
\begin{DoxyCode}{0}
\DoxyCodeLine{clean:}
\DoxyCodeLine{    rm *.o}
\DoxyCodeLine{vector: }
\DoxyCodeLine{    g++ -\/c function.cpp \&\& g++ -\/o vector vector.cpp function.o}
\DoxyCodeLine{list: }
\DoxyCodeLine{    g++ -\/c function.cpp \&\& g++ -\/o list list.cpp function.o}
\DoxyCodeLine{deque: }
\DoxyCodeLine{    g++ -\/c function.cpp \&\& g++ -\/o deque deque.cpp function.o}

\end{DoxyCode}

\end{DoxyItemize}

Testavimo sistemos parametrai\+:
\begin{DoxyItemize}
\item CPU\+: Intel i5 8250U (4 cores, 8 threads)
\item RAM\+: 12GB DDR4
\item SSD\+: Hynix 256GB SATA 3.\+0 (OS)
\item SSD\+: Crucial 1TB SATA 3.\+0 
\end{DoxyItemize}